\documentclass[a4paper,UKenglish,cleveref, autoref, thm-restate]{lipics-v2019}
%This is a template for producing LIPIcs articles. 
%See lipics-manual.pdf for further information.
%for A4 paper format use option "a4paper", for US-letter use option "letterpaper"
%for british hyphenation rules use option "UKenglish", for american hyphenation rules use option "USenglish"
%for section-numbered lemmas etc., use "numberwithinsect"
%for enabling cleveref support, use "cleveref"
%for enabling autoref support, use "autoref"
%for anonymousing the authors (e.g. for double-blind review), add "anonymous"
%for enabling thm-restate support, use "thm-restate"

%\graphicspath{{./graphics/}}%helpful if your graphic files are in another directory

\bibliographystyle{plainurl}% the mandatory bibstyle

\title{Decidability results of Communicating Finite
State Machines over acyclic topology } %TODO Please add

\titlerunning{} %TODO optional, please use if title is longer than one line

\author{G. Namratha Reddy}{Chennai Mathematical Institute, India \and  \url{http://www.cmi.ac.in/~namratha} }{}{}{}

% \authorrunning{J.\,Q. Public and J.\,R. Public} %TODO mandatory. First: Use abbreviated first/middle names. Second (only in severe cases): Use first author plus 'et al.'

% \Copyright{John Q. Public and Joan R. Public} %TODO mandatory, please use full first names. LIPIcs license is "CC-BY";  http://creativecommons.org/licenses/by/3.0/

% \ccsdesc[100]{\textcolor{red}{Replace ccsdesc macro with valid one}} %TODO mandatory: Please choose ACM 2012 classifications from https://dl.acm.org/ccs/ccs_flat.cfm 

\keywords{Distributed Systems, Reachability, Finite state machines with FIFO queues, Network of concurrent processes} %TODO mandatory; please add comma-separated list of keywords

\category{} %optional, e.g. invited paper

\relatedversion{} %optional, e.g. full version hosted on arXiv, HAL, or other respository/website
%\relatedversion{A full version of the paper is available at \url{...}.}

\supplement{}%optional, e.g. related research data, source code, ... hosted on a repository like zenodo, figshare, GitHub, ...

%\funding{(Optional) general funding statement \dots}%optional, to capture a funding statement, which applies to all authors. Please enter author specific funding statements as fifth argument of the \author macro.

\acknowledgements{}%optional nam-maybe-later

%\nolinenumbers %uncomment to disable line numbering

%\hideLIPIcs  %uncomment to remove references to LIPIcs series (logo, DOI, ...), e.g. when preparing a pre-final version to be uploaded to arXiv or another public repository

%Editor-only macros:: begin (do not touch as author)%%%%%%%%%%%%%%%%%%%%%%%%%%%%%%%%%%
% \EventEditors{John Q. Open and Joan R. Access}
% \EventNoEds{2}
% \EventLongTitle{42nd Conference on Very Important Topics (CVIT 2016)}
% \EventShortTitle{CVIT 2016}
% \EventAcronym{CVIT}
% \EventYear{2016}
% \EventDate{December 24--27, 2016}
% \EventLocation{Little Whinging, United Kingdom}
% \EventLogo{}
% \SeriesVolume{42}
% \ArticleNo{23}
%%%%%%%%%%%%%%%%%%%%%%%%%%%%%%%%%%%%%%%%%%%%%%%%%%%%%%

\begin{document}

\maketitle

%TODO mandatory: add short abstract of the document
\begin{abstract}
    The paper shows that the reachability problem for concurrent finite state processes which communicate over FIFO queues is decidable over acyclic topologies. 
\end{abstract}

\section{Introduction}
\label{sec:typesetting-summary}

LIPIcs is a series of open access high-quality conference proceedings across all fields in informatics established in cooperation with Schloss Dagstuhl. 
In order to do justice to the high scientific quality of the conferences that publish their proceedings in the LIPIcs series, which is ensured by the thorough review process of the respective events, we believe that LIPIcs proceedings must have an attractive and consistent layout matching the standard of the series.
Moreover, the quality of the metadata, the typesetting and the layout must also meet the requirements of other external parties such as indexing service, DOI registry, funding agencies, among others. The guidelines contained in this document serve as the baseline for the authors, editors, and the publisher to create documents that meet as many different requirements as possible. 

Please comply with the following instructions when preparing your article for a LIPIcs proceedings volume. 
\paragraph*{Minimum requirements}

\begin{itemize}
\item Use pdflatex and an up-to-date \LaTeX{} system.
\item Use further \LaTeX{} packages and custom made macros carefully and only if required.
\item Use the provided sectioning macros: \verb+\section+, \verb+\subsection+, \verb+\subsubsection+, \linebreak \verb+\paragraph+, \verb+\paragraph*+, and \verb+\subparagraph*+.
\item Provide suitable graphics of at least 300dpi (preferably in PDF format).
\item Use BibTeX and keep the standard style (\verb+plainurl+) for the bibliography.
\item Please try to keep the warnings log as small as possible. Avoid overfull \verb+\hboxes+ and any kind of warnings/errors with the referenced BibTeX entries.
\item Use a spellchecker to correct typos.
\end{itemize}

\section{System topology}

\subsection{Model}

We work with a topology which we describe using the tuple $(P, C, Reader, Writer)$ where $P$ is the set of processes, $C$ is the set of channels, $Reader$ is a function $C \rightarrow P$ and $Writer$ is a function $C \rightarrow P$. 
Also, we assume that no channel has the same reader and writer.


Each process is a finite state transition system $TS_i = (Q_i, \Sigma_i, \delta_i, s_i)$

%nam-rephrase-this
Since we are interested only in control state reachability and not in language
theory, we will ignore the alphabet and final states. However the transitions
may also have associated actions, which in our case may be sending a mes-
sage to a channel or receiving a message from a channel. 


% describe what delta_i looks like
$\delta_i$ has q -> q' on  c?a if is a reader, c!a if it is a writer or a nop


Each channel holds messages coming from a finite set i.e. $c_i \in C$ can have messages from $M_i$ 

\subsection{Configuration Graph}

A global configuration of the system would consist of the states of each process and the channel contents of each channel. So if we have n processes and m channels, a configuration would be the tuple $(q_0, q_1, ..., q_n, \gamma_1, \gamma_2, ..., \gamma_m)$


% decsribe how the edges of the global transition graph work

that is when a transition from one tuple to the next happens what all should align in the universe.

A run is a path in this graph 


\section{The Reachability Problem}

The reachability problem is to ask whether we can reach a target configuration where one or more processes are in a target state.

We know that the problem is undecidable in general (cite), because if there is a loop in the topology then we can simulate a queue machine and state reachability is undecidable for queue machines. 

So we ask the question of whether it is decidable for acyclic topologies.

What we mean by acyclic topologies is consider the network topology and ignore the direction of the edges, so we get an undirected graph and this graph should have no cycles.

We solve this with the help of two reductions

\paragraph*{Reduction 1}

given such an acyclic topology the reachability problem can be reduced to one in which the target state is reached only if the queue is empty (cite Madhu)

\paragraph*{Reduction 2}
We reduce it to another isomorphic topology that looks like a tree, where every process has one incoming edge (except the root) i.e every process can read from one channel but write to multiple channels (cite Madhu)


\begin{theorem}\label{testenv-theorem}
 The Reachability problem is decidable for CFMs with FIFO channels over undirected topology. 
\end{theorem}


\begin{proof}
 We use reductions 1 and 2 to get a tree topology
 
\end{proof}
    

Nec urna malesuada sollicitudin. Nulla facilisi. Vivamus aliquam tempus ligula eget ornare. Praesent eget magna ut turpis mattis cursus. Aliquam vel condimentum orci. Nunc congue, libero in gravida convallis \cite{DBLP:conf/focs/HopcroftPV75}, orci nibh sodales quam, id egestas felis mi nec nisi. Suspendisse tincidunt, est ac vestibulum posuere, justo odio bibendum urna, rutrum bibendum dolor sem nec tellus. 

\begin{lemma} [Quisque blandit tempus nunc]
Sed interdum nisl pretium non. Mauris sodales consequat risus vel consectetur. Aliquam erat volutpat. Nunc sed sapien ligula. Proin faucibus sapien luctus nisl feugiat convallis faucibus elit cursus. Nunc vestibulum nunc ac massa pretium pharetra. Nulla facilisis turpis id augue venenatis blandit. Cum sociis natoque penatibus et magnis dis parturient montes, nascetur ridiculus mus.
\end{lemma}

Fusce eu leo nisi. Cras eget orci neque, eleifend dapibus felis. Duis et leo dui. Nam vulputate, velit et laoreet porttitor, quam arcu facilisis dui, sed malesuada risus massa sit amet neque.

\section{Conclusions}

Morbi eros magna, vestibulum non posuere non, porta eu quam. Maecenas vitae orci risus, eget imperdiet mauris. Donec massa mauris, pellentesque vel lobortis eu, molestie ac turpis. Sed condimentum convallis dolor, a dignissim est ultrices eu. Donec consectetur volutpat eros, et ornare dui ultricies id. Vivamus eu augue eget dolor euismod ultrices et sit amet nisi. Vivamus malesuada leo ac leo ullamcorper tempor. Donec justo mi, tempor vitae aliquet non, faucibus eu lacus. Donec dictum gravida neque, non porta turpis imperdiet eget. Curabitur quis euismod ligula. 


%%
%% Bibliography
%%

%% Please use bibtex, 

\bibliography{lipics-v2019-sample-article}

\appendix

\section{Styles of lists, enumerations, and descriptions}\label{sec:itemStyles}

List of different predefined enumeration styles:

\begin{itemize}
\item \verb|\begin{itemize}...\end{itemize}|
\item \dots
\item \dots
%\item \dots
\end{itemize}

\begin{enumerate}
\item \verb|\begin{enumerate}...\end{enumerate}|
\item \dots
\item \dots
%\item \dots
\end{enumerate}

\begin{alphaenumerate}
\item \verb|\begin{alphaenumerate}...\end{alphaenumerate}|
\item \dots
\item \dots
%\item \dots
\end{alphaenumerate}

\begin{romanenumerate}
\item \verb|\begin{romanenumerate}...\end{romanenumerate}|
\item \dots
\item \dots
%\item \dots
\end{romanenumerate}

\begin{bracketenumerate}
\item \verb|\begin{bracketenumerate}...\end{bracketenumerate}|
\item \dots
\item \dots
%\item \dots
\end{bracketenumerate}

\begin{description}
\item[Description 1] \verb|\begin{description} \item[Description 1]  ...\end{description}|
\item[Description 2] Fusce eu leo nisi. Cras eget orci neque, eleifend dapibus felis. Duis et leo dui. Nam vulputate, velit et laoreet porttitor, quam arcu facilisis dui, sed malesuada risus massa sit amet neque.
\item[Description 3]  \dots
%\item \dots
\end{description}

\cref{testenv-proposition} and \autoref{testenv-proposition} ...

\section{Theorem-like environments}\label{sec:theorem-environments}

List of different predefined enumeration styles:

\begin{theorem}\label{testenv-theorem}
Fusce eu leo nisi. Cras eget orci neque, eleifend dapibus felis. Duis et leo dui. Nam vulputate, velit et laoreet porttitor, quam arcu facilisis dui, sed malesuada risus massa sit amet neque.
\end{theorem}

\begin{lemma}\label{testenv-lemma}
Fusce eu leo nisi. Cras eget orci neque, eleifend dapibus felis. Duis et leo dui. Nam vulputate, velit et laoreet porttitor, quam arcu facilisis dui, sed malesuada risus massa sit amet neque.
\end{lemma}

\begin{corollary}\label{testenv-corollary}
Fusce eu leo nisi. Cras eget orci neque, eleifend dapibus felis. Duis et leo dui. Nam vulputate, velit et laoreet porttitor, quam arcu facilisis dui, sed malesuada risus massa sit amet neque.
\end{corollary}

\begin{proposition}\label{testenv-proposition}
Fusce eu leo nisi. Cras eget orci neque, eleifend dapibus felis. Duis et leo dui. Nam vulputate, velit et laoreet porttitor, quam arcu facilisis dui, sed malesuada risus massa sit amet neque.
\end{proposition}

\begin{exercise}\label{testenv-exercise}
Fusce eu leo nisi. Cras eget orci neque, eleifend dapibus felis. Duis et leo dui. Nam vulputate, velit et laoreet porttitor, quam arcu facilisis dui, sed malesuada risus massa sit amet neque.
\end{exercise}

\begin{definition}\label{testenv-definition}
Fusce eu leo nisi. Cras eget orci neque, eleifend dapibus felis. Duis et leo dui. Nam vulputate, velit et laoreet porttitor, quam arcu facilisis dui, sed malesuada risus massa sit amet neque.
\end{definition}

\begin{example}\label{testenv-example}
Fusce eu leo nisi. Cras eget orci neque, eleifend dapibus felis. Duis et leo dui. Nam vulputate, velit et laoreet porttitor, quam arcu facilisis dui, sed malesuada risus massa sit amet neque.
\end{example}

\begin{note}\label{testenv-note}
Fusce eu leo nisi. Cras eget orci neque, eleifend dapibus felis. Duis et leo dui. Nam vulputate, velit et laoreet porttitor, quam arcu facilisis dui, sed malesuada risus massa sit amet neque.
\end{note}

\begin{note*}
Fusce eu leo nisi. Cras eget orci neque, eleifend dapibus felis. Duis et leo dui. Nam vulputate, velit et laoreet porttitor, quam arcu facilisis dui, sed malesuada risus massa sit amet neque.
\end{note*}

\begin{remark}\label{testenv-remark}
Fusce eu leo nisi. Cras eget orci neque, eleifend dapibus felis. Duis et leo dui. Nam vulputate, velit et laoreet porttitor, quam arcu facilisis dui, sed malesuada risus massa sit amet neque.
\end{remark}

\begin{remark*}
Fusce eu leo nisi. Cras eget orci neque, eleifend dapibus felis. Duis et leo dui. Nam vulputate, velit et laoreet porttitor, quam arcu facilisis dui, sed malesuada risus massa sit amet neque.
\end{remark*}

\begin{claim}\label{testenv-claim}
Fusce eu leo nisi. Cras eget orci neque, eleifend dapibus felis. Duis et leo dui. Nam vulputate, velit et laoreet porttitor, quam arcu facilisis dui, sed malesuada risus massa sit amet neque.
\end{claim}

\begin{claim*}\label{testenv-claim2}
Fusce eu leo nisi. Cras eget orci neque, eleifend dapibus felis. Duis et leo dui. Nam vulputate, velit et laoreet porttitor, quam arcu facilisis dui, sed malesuada risus massa sit amet neque.
\end{claim*}

\begin{proof}
Fusce eu leo nisi. Cras eget orci neque, eleifend dapibus felis. Duis et leo dui. Nam vulputate, velit et laoreet porttitor, quam arcu facilisis dui, sed malesuada risus massa sit amet neque.
\end{proof}

\begin{claimproof}
Fusce eu leo nisi. Cras eget orci neque, eleifend dapibus felis. Duis et leo dui. Nam vulputate, velit et laoreet porttitor, quam arcu facilisis dui, sed malesuada risus massa sit amet neque.
\end{claimproof}

\end{document}
