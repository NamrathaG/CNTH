\documentclass{article}
\usepackage[utf8]{inputenc}
\usepackage{amsmath, amssymb, mathrsfs}
\usepackage{array}
\usepackage[body={5.5in,8.75in}]{geometry}

\title{Problem Set 8}
\date{}
\begin{document}

\maketitle


Given the reachability problem i.e the configuration of states that need to be reached in a subset of process. We define the below languages
Let $P \in \mathscr{P}$ be the set of processes whose target state reachability we are interested in. 
We define the following automaton for them using their transition systems.
For every $i \in \mathscr{P}$ we define $A_i = (Q_i, \Sigma_i, \delta_i, s_i, F_i)$

where $Q_i$ are the same as in the transition system, $\Sigma_i$ = union of channel alphabets for which this process is a reader and writer,
$\delta_i$ is the same as in the transition system but the operation replaced by the message letter being read or written and no-op replaced by $\epsilon$ 
$s_i$ is the same as transition system initial state
$F_i$ is the target state we are interested in if $i \in P$ else it is $Q_i$

$L_i$ is the language accepted by the automaton $A_i$

ASSUMPTION : w.l.o.g we assume that all the channel alphabets are disjoint


Given the directed tree topology, we define a language $L_i^e$ 
If process i is a leaf then $L_i^e = L_i \cap \Sigma_{i_r}^*$ where $i_r$ is the channel process $i$ reads from.
If process i is a non-leaf, and the children of i are $k_1, k_2, ..., k_m$ then $L_i^e = L_i \cap shuffle(L_{k_1}^e, L_{k_2}^e, ..., L_{k_m}^e)$



\end{document}