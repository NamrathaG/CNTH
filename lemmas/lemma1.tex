\documentclass{article}
\usepackage[utf8]{inputenc}
\usepackage{amsmath, amssymb, mathrsfs}
\usepackage{array}
\usepackage[body={5.5in,8.75in}]{geometry}

\title{Problem Set 8}
\date{}
\begin{document}

\maketitle


Because we have a directed tree topology, we have for each node i r(i), and for consistency let's assume the root reads from a special channel which is empty 
this way every node reads from some channel


ASSUMPTION: For simplicity lets define reachability such that we are interested in every process reaching a target state

Given the reachability problem i.e the configuration of states that need to be reached in a subset of process. We define the below languages
Let $P \in \mathscr{P}$ be the set of processes whose target state reachability we are interested in. 
We define the following automaton for them using their transition systems.
For every $i \in \mathscr{P}$ we define $A_i = (Q_i, \Sigma_i, \delta_i, s_i, F_i)$

where $Q_i$ are the same as in the transition system, $\Sigma_i$ = union of channel alphabets for which this process is a reader and writer,
$\delta_i$ is the same as in the transition system but the operation replaced by the message letter being read or written and no-op replaced by $\epsilon$ 
$s_i$ is the same as transition system initial state
$F_i$ is the target state we are interested in if $i \in P$ else it is $Q_i$

$L_i$ is the language accepted by the automaton $A_i$

$L_i\!\!\!\downarrow_r$ is the language where only symbols from the reader channel are present, i.e other alphabets are replaced by epsilon. 

ASSUMPTION : w.l.o.g we assume that all the channel alphabets are disjoint
ASSUMPTION : additionally lets say we removed all the epsilon transitions and have a DFA

After we do all of this a run it is clear that we can go from a run in the global configuration graph to a run in this system, basically the channels are removed that's all right. 


But note that a particular transition involving an alphabet from reader channel for that process is only possible if that alphabet was at the head in the reader channel

Given the directed tree topology, we define a language $L_i^e$ as follows:
If process i is a leaf then $L_i^e = L_i \cap M_{r(i)}^*$ where $r(i)$ is the channel that process $i$ reads from.

If process i is a non-leaf, and the children of i are $k_1, k_2, ..., k_m$ then $L_i^e = L_i \cap shuffle(L_{k_1}^e\!\!\!\downarrow_{r}, L_{k_2}^e\!\!\!\downarrow_{r}, ..., L_{k_m}^e\!\!\!\downarrow_{r}, M_{r(i)}^*)$


\section{Lemma1}

% Claim: Reachable(there exists a run such that the target states are reached in the given topology) iff $L_r^e$ is not empty (Where r is the root process in the topology) 


% ($\leftarrow$)
% Induction on the number of nodes in the tree

% Base case: there is only one node
% $L_r^e$ is $\epsilon$ which is non empty, imples $\epsilon \in L_r$, implies $s_r = f_r$, clearly since we start in $s_r$ anyway it is reachable


% Induction : Let root be a non-leaf node and let the children of root be and the children of i are $k_1, k_2, ..., k_m$,  $L_r^e$ is non-empty implies $L_i^e = L_i \cap shuffle(L_{k_1}^e\!\!\!\downarrow_{r}, L_{k_2}^e\!\!\!\downarrow_{r}, ..., L_{k_m}^e\!\!\!\downarrow_{r}, M_{r(i)}^*)$ is non-empty

% this is non-empty implies all the children ones are also non-empty

% By induction hypothesis 



% Claim1: If there is a run rho from (s0,s1,...sn) to (f0,f1,...fn) implies project rho to root process, the labels of the run form a word w in $L_r^e$


% Proof1: 







Consider the tree topology

Lemma1 : For any word w in $L_j^e$ with channel r(i) supplying the $w \downarrow_{r(i)}$ then  we can form a run from (s0,s1,...sn) to (f0,f1,...fn) where {0...n} are the nodes in the subtree rooted by j

Proof2: Induction on the no. of nodes in the directed tree topology

Base case: $L_j^e$ is $\epsilon$ which is non empty, imples $\epsilon \in L_j$, implies $s_j = f_j$, clearly since we start in $s_j$ anyway it is reachable

Induction: 
Let i be a non-leaf node and let the children be $k_1, k_2, ..., k_m$,  $L_r^e$ is non-empty implies $L_i^e = L_i \cap shuffle(L_{k_1}^e\!\!\!\downarrow_{r}, L_{k_2}^e\!\!\!\downarrow_{r}, ..., L_{k_m}^e\!\!\!\downarrow_{r}, M_{r(i)}^*)$ is non-empty


let $w \in L_r^e$
$w$ is of the form $shuffle(w_1, w_2, ..., w_m)$

$w_i\downarrow_{r(k_i)}$ = $w_i'\downarrow{r(k_i)}$ where $w_i' \in L_{k_i}^e$


By induction hypothesis if channel $r(k_i)$ supplies the word $w_i'\downarrow{r(k_i)}$ then we can generate run where nodes in the tree rooted by $k_i$ all reach their target states. 

Since $w \in L_r^e$ does supply it, we can form this run. 
Similarly we can form runs for nodes in subtrees rooted at other children


By concatenating the runs we can form the final run that we want.



Theorem 1:  Reachable(there exists a run such that the target states are reached in the given topology) iff $L_r^e$ is not empty (Where r is the root process in the topology) 

($\leftarrow$) Since root does not need anything to be supplied in its channel, by using lemma 1

($\rightarrow$) 





\section{Lemma 2}



Claim 1: Given a directed tree topology with r=0 as the root of the tree.


$w \in L_r^e \implies (s_0, s_1, ...,s_n, w \downarrow_{r(0)}, \epsilon, ...,\epsilon) \rightarrow_G^* (f_0, f_1, ..., f_n, \epsilon, \epsilon, ... \epsilon)$

Proof1: 
Induction on the no. of nodes in the directed tree topology

Base case: We have only one node r,  $L_r^e = {\epsilon} \implies \epsilon \in L_r \implies s_r = f_r$, clearly since we start in $s_r$ it is reachable

Induction: 
Let r be a non-leaf node and let the children be $k_1, k_2, ..., k_m$,  
$w \in L_r^e \implies w \in L_i \cap shuffle(L_{k_1}^e\!\!\!\downarrow_{r}, L_{k_2}^e\!\!\!\downarrow_{r}, ..., L_{k_m}^e\!\!\!\downarrow_{r}, M_{r(i)}^*)$
$w$ is of the form $shuffle(w_1, w_2, ..., w_m)$ where $w_i \in L_{k_i}^e\!\!\!\downarrow_{r(i)}$

$w_i\downarrow_{r(k_i)}$ = $w_i'\downarrow{r(k_i)}$ where $w_i' \in L_{k_i}^e$

Let $\pi_i$ be the tree that is rooted by $k_i$
By induction hypothesis, since $w_i' \in L_{k_i}^e$ we have run from nodes in $\pi$ are in start state, and reader channel of $k_i$ has $w_i'\downarrow{r(k_i)}$ 
to a final configuration where every node is in the final state and all channels are empty.

Since w is in $L_r^e$ it means it is also in $L_r$ we can take this run and convert to the corresponding run in the configuration graph where only r is moving, 
and this w can generate $w_i'\downarrow{r(k_i)}$ in each $k_i$'s reader channel, so we can start from $(s_0, s_1,...,s_n, \epsilon ...) -> (\text{only r moves and fills up all the channels}) -> (f_0, s_1, ..., s_n,\epsilon, w_i\downarrow_{r(k_i)}, ... w_i\downarrow_{r(k_i)} ) -> stitch the k_i run one after the other$ 

% By induction hypothesis if channel $r(k_i)$ supplies the word $w_i'\downarrow{r(k_i)}$ then we can generate run where nodes in the tree rooted by $k_i$ all reach their target states. 

% Since $w \in L_r^e$ does supply it, we can form this run. 
% Similarly we can form runs for nodes in subtrees rooted at other children


% By concatenating the runs we can form the final run that we want.






Claim 2: Given a directed tree topology with i=0 as the root of the tree.

$(s_0, s_1, ...,s_n, \alpha, \epsilon, ...,\epsilon) \rightarrow_G^* (f_0, f_1, ..., f_n, \epsilon, \epsilon, ... \epsilon) => \exists w \in L_i^e,   \alpha \downarrow_{r(0)} = w \downarrow_{r(0)}$


Proof2:

Base case : There is only one node $\alpha$ has to be epsilon 

Induction: 
Let r be a non-leaf node, and $k_1, k_2, ..., k_n$ be the children of the r and let the tree rooted by them be $\pi_1, \pi_2, .., \pi_n$ 

Let r run and fill all the reader channels with $\alpha_i$
For each $k_i$ we form a run from where only nodes in $\pi_i$ are in start state and $\alpha_i$ is in the reader channel of $k_i$, there and final state is reached. Now we use induction hypothesis to say that there is a word in $w_i \in L_{k_i}^e$ and $\alpha_i \downarrow_{r(0)} = w_i \downarrow_{r(0)}$

So we find a $w_i$ for each $k_i$

so we have $w_i \downarrow_{r(i)} \in L_i^e$ 
so we have  $\alpha_i \in L_i^e$ 

Consider the word $w \in shuffle(\alpha_1, \alpha_2,...\alpha_n, \alpha)$
$w also belongs to L_r$

$w \in L_r^e$ and $w \downarrow_{r(r)} = \alpha $ 





\end{document}